\clearpage

\section{Grading guidelines}

\subsubsection{Written Work}

An \textbf{A} or \textbf{A-} thesis, paper, or exam is one that is good
enough to be read aloud in a class. It is clearly written and
well-organized. It demonstrates that the writer has conducted a close
and critical reading of texts, grappled with the issues raised in the
course, synthesized the readings, discussions, and lectures, and
formulated a perceptive, compelling, independent argument. The argument
shows intellectual originality and creativity, is sensitive to
historical context, is supported by a well-chosen variety of specific
examples, and, in the case of a research paper, is built on a critical
reading of primary material.

A \textbf{B+} or \textbf{B} thesis, paper, or exam demonstrates many
aspects of A-level work but falls short of it in either the organization
and clarity of its writing, the formulation and presentation of its
argument, or the quality of research. Some papers or exams in this
category are solid works containing flashes of insight into many of the
issues raised in the course. Others give evidence of independent
thought, but the argument is not presented clearly or convincingly.

A \textbf{B-} thesis, paper, or exam demonstrates a command of course or
research material and understanding of historical context but provides a
less than thorough defense of the writer's independent argument because
of weaknesses in writing, argument, organization, or use of evidence.

A \textbf{C+}, \textbf{C}, or \textbf{C-} thesis, paper, or exam offers
little more than a mere a summary of ideas and information covered in
the course, is insensitive to historical context, does not respond to
the assignment adequately, suffers from frequent factual errors, unclear
writing, poor organization, or inadequate primary research, or presents
some combination of these problems.

Whereas the grading standards for written work between A and C- are
concerned with the presentation of argument and evidence, a paper or
exam that belongs to the D or F categories demonstrates inadequate
command of course material.

A \textbf{D} thesis, paper, or exam demonstrates serious deficiencies or
severe flaws in the student's command of course or research material.

An \textbf{F} thesis, paper, or exam demonstrates no competence in the
course or research materials. It indicates a student's neglect or lack
of effort in the course.

\subsubsection{Discussions and Seminars}

A student who receives an \textbf{A} for participation in discussion in
discussions or seminars typically comes to every class with questions
about the readings in mind. An `A' discussant engages others about
ideas, respects the opinions of others, and consistently elevates the
level of discussion.

A student who receives a \textbf{B} for participation in discussion in
discussions or seminars typically does not always come to class with
questions about the readings in mind. A `B' discussant waits passively
for others to raise interesting issues. Some discussants in this
category, while courteous and articulate, do not adequately listen to
other participants or relate their comments to the direction of the
conversation.

A student who receives a \textbf{C} for discussion in discussions or
seminars attends regularly but typically is an infrequent or unwilling
participant in discussion.

A student who fails to attend discussions or seminars regularly and
adequately prepared for discussion risks the grade of \textbf{D} or
\textbf{F}.

--Taken from
\href{http://www.princeton.edu/history/undergraduate/grading\_practices/}{the
department of history at Princeton University.}
