\documentclass[12pt,twoside]{article}

%All course-specific variables are defined in a .cnf file for the course.

\newcommand{\InstructorName}{Benjamin Schmidt}
\newcommand{\Institution}{Northeastern University}
\newcommand{\CourseNumber}{HONR 1205-3}
\newcommand{\CourseTitle}{The History of Big Data}
\newcommand{\OfficeHours}{Tuesdays 3-5pm, 219 Meserve Hall; or by appointment}
\newcommand{\Email}{bschmidt@neu.edu}
\newcommand{\Semester}{Fall 2013}
\newcommand{\ZoteroLibraryCode}{3G43MVHQ}

\usepackage[english]{babel}
\usepackage{fontspec}

% Font choice:
%
% Garamond Premier Pro, unfortunately, costs money.
%
% Any owner of Adobe Reader actually has copies of the excellent, quite
% full-featured OpenType font Minion Pro. Look in Reader's application
% files.
%
% On a Mac, Hoefler Text makes a reasonable fallback.
%
% Finally, there are some free Palatino variants. For use with xelatex,
% I have been okay with TeX Gyre Pagella (an open-source font). If
% you wish, you could remove the xelatex/fontspec dependency here and
% instead use pdflatex with package mathpazo.

\setmainfont[Ligatures=TeX,Numbers=OldStyle,%
    BoldFont={* Bold}]{Minion Pro}

\usepackage{xunicode}
\usepackage{xltxtra} 

\usepackage{hanging}
\usepackage{fancyhdr}
\usepackage{geometry}
\usepackage{setspace}
\usepackage{xkeyval}
\usepackage[bf,small,raggedright,compact]{titlesec}
\usepackage{enumerate}

% This file, with version control info, is generated by running ./vc 
%%% This file has been generated by the vc bundle for TeX.
%%% Do not edit this file!
%%%
%%% Define Git specific macros.
\gdef\GITHash{4d2fcdc81f399bda4ffa7209c943699ff51f26c4}%
\gdef\GITAbrHash{4d2fcdc}%
\gdef\GITParentHashes{}%
\gdef\GITAbrParentHashes{}%
\gdef\GITAuthorName{Benjamin Schmidt}%
\gdef\GITAuthorEmail{bmschmidt@gmail.com}%
\gdef\GITAuthorDate{2013-08-26 23:10:17 -0400}%
\gdef\GITCommitterName{Benjamin Schmidt}%
\gdef\GITCommitterEmail{bmschmidt@gmail.com}%
\gdef\GITCommitterDate{2013-08-26 23:10:17 -0400}%
%%% Define generic version control macros.
\gdef\VCRevision{\GITAbrHash}%
\gdef\VCAuthor{\GITAuthorName}%
\gdef\VCDateRAW{2013-08-26}%
\gdef\VCDateISO{2013-08-26}%
\gdef\VCDateTEX{2013/08/26}%
\gdef\VCDateUSA{08/26/2013}%
\gdef\VCTime{23:10:17 -0400}%
\gdef\VCModifiedText{\textcolor{red}{with local modifications!}}%
%%% Assume clean working copy.
\gdef\VCModified{0}%
\gdef\VCRevisionMod{\VCRevision}%


% Some tweaks necessary to ensure annotations print at the end of 
% bibliography entries.
\usepackage{csquotes}
\usepackage[notes,annotation,short,backend=biber]{biblatex-chicago}
\DeclareFieldFormat{annotation}{#1\isdot}
\renewcommand{\bibfont}{\footnotesize}
\bibliography{course.bib}

\setcounter{secnumdepth}{-2}	% Suppress section numbers even with unstarred
				% (sub)section commands.

% adjust margins as you will
\geometry{hcentering=true,xetex}

% page headers. Set up for headers on odd-side pages only
\pagestyle{fancy}

\renewcommand{\footrulewidth}{0 pt} % I don't like fancyhdr's rules.
\renewcommand{\headrulewidth}{0 pt}
\fancyhead{}
\fancyhead[LO]{\small \CourseNumber{}}
\fancyhead[CO]{\small \Institution{}}
\fancyhead[RO]{\small \Semester{}}
\fancyfoot{}
\cfoot{\small \thepage}
% VCDateUSA macro supplied in tweaked vc-git.awk
\rfoot{\small Last revised \VCDateUSA}

% extra leading rather than indents to separate paragraphs
\singlespacing
\setlength{\parindent}{0 pt}
\setlength{\parskip}{0.25\baselineskip}

% overfull hboxes, begone
\setlength{\emergencystretch}{2 em}

\usepackage[dvipsnames]{xcolor}
\usepackage[
  pdftitle={\CourseTitle{}},
  pdfauthor=\InstructorName{},
  bookmarks, bookmarksopen,
  colorlinks=true,urlcolor=blue,citecolor=BlueViolet,
  xetex]{hyperref}
\urlstyle{same}

\begin{document}

% Sorry, world, but it's amazing how annoying it is to convince the
% titling package to do what I want with titles on documents like these.
%
% So no \maketitle.

\begin{flushleft}
\textbf{\CourseTitle{} \\
Professor \InstructorName} \\
%\url{\Website} \\
Email: \Email \\
Office Hours: \OfficeHours
\end{flushleft}


\section{Overview}

This course helps students to understand contemporary excitement and
fears about ``Big Data'' in a long historical context. Much is new about
the way corporations, governments, and individuals use massive
computational resources to search for patterns. But those who use big
data draw on legacies from well before the computer age for data
management, for structuring a complicated world into measurable
quantities.

\subsection{Course Themes}

We will trace the long history of big data through four parallel
strands:

\begin{enumerate}
\def\labelenumi{\arabic{enumi}.}
\item
  The rise of massive systems of data collection by the government in
  the 19th century through institutions like the census and the
  military.
\item
  The attempts of American businesses to collect and use data to control
  their markets and their workers.
\item
  The turn to data by the sciences.
\item
  The development of computers from the 1940s on, and the ways that
  social forces shaped the development of computing.
\end{enumerate}

\section{Course Goals}

\begin{enumerate}
\def\labelenumi{\arabic{enumi}.}
\itemsep1pt\parskip0pt\parsep0pt
\item
  This course will teach students the long history of statistical
  thinking in the United States
\item
  It will give a grounding in critical approaches
\item
  It will give an introduction to methods for
\end{enumerate}

\section{Requirements}

\paragraph{Classroom}

You must complete all the readings for the course and attend class
prepared to discuss them.

\paragraph{Response Posts to Blackboard}

6 times in the semester, you will post a short \emph{response} to one or
more of the readings for that day on Blackboard. These must be posted by
5pm the day \emph{before} class meets so that your peers have time to
read them. You must also write 6 \emph{responses} to your peer's posts.
Both posts and responses will be included as part of your participation
grade.

\paragraph{Papers}

You will write one 5 to 7 page paper for this class, based on the
readings; no outside research is expected.

\paragraph{Archival Project}

Mid-semester, we will take a trip to the university archives to look at
some archival documents. You will write up a description of another
document from elsewhere in the archives. This can take the form of a
straightforward narration, or you can adapt the information

\paragraph{Final Project}

Final project assignments will be distributed in October, but you should
start thinking early about which one you will want. It will consist of
either 1) an 8-10 page paper in which you extend one of the weeks of the
course with additional readings; or 2) a digital project in which you
analyze a dataset created before the year 1994 using modern tools. In
either case, you must discuss the project in advance with me.

\section{Academic Integrity}

You are expected to have read, and follow at all times, the University's
\href{http://www.northeastern.edu/osccr/academicintegrity/index.html}{Academic
Integrity Policy}.

\section{Grading}

\section{Schedule and Readings}

\subsection{Required for purchase: these may be placed in the University
bookstore, but you will probably be better off ordering used or new
copies online.}

\begin{itemize}
\item
  Frederick Taylor, \emph{The Principles of Scientific Management}
\item
  Sarah Igo, \emph{The Averaged American}
\item
  Nicholas Lemann, \emph{The Big Test}
\item
  Tracy Kidder, \emph{The Soul of a New Machine}
\item
  Tim Berners-Lee, \emph{Weaving the Web}
\end{itemize}

All other readings will be made available through Blackboard.

In addition to the readings listed below, some short primary sources may
be added to Blackboard as the course develops to accompany a week's
readings.

\subsection{Week 1: Introductions}

\subsection{Week 2: Early Modern Information Overload}

\subsubsection{September 9}

\begin{itemize}
\item
  \cite{blair_reading_2003}
\item
  (In class): Plato, \emph{Phaedrus}, on the invention of writing.
\end{itemize}

\subsubsection{September 12}

\begin{itemize}
\item
  \cite{muller-wille_natural_2012}
\end{itemize}

\subsection{Week 3: Ordering the World}

\subsubsection{September 16}

\begin{itemize}
\item
  \cite{borges_analytical_1999}
\item
  \cite[Introduction and Chapter 3]{foucault_order_1994}
\end{itemize}

\subsubsection{September 19}

\begin{itemize}
\item
  Try the Foucault again
\item
  In class: \cite{playfair_commercial_2005}
\end{itemize}

\subsection{Week 4: 19th Century Records}

\subsubsection{September 23: Accounting}

\begin{itemize}
\item
  \cite{edwards_early_1960}
\end{itemize}

\subsubsection{September 26: Slave economies}

\begin{itemize}
\item
  \cite{garvey_facts_2013}
\item
  \cite[pp. 192-209,233-258]{fogel_time_1989}
\item
  \cite["GW Hammond, Instructions to his Overseer"]{wilentz_major_1992}
\end{itemize}

\subsection{Week 5: Managed Information in the late nineteenth century}

\subsubsection{September 30: Industrial Revolutions}

\begin{itemize}
\item
  \cite[Chapter 6, "Industrial Revolution and the Crisis of Control"]{beniger_control_1986}
\item
  In Class: Dun and Bradstreet. \cite{t._mercantile_1851},
  \cite{j._mercantile_1851}
\end{itemize}

\subsubsection{October 3: The Census}

\begin{itemize}
\item
  \cite{kinnahan_charting_2008}
\item
  \cite[Chapters 3 to 5]{anderson_american_1988}
\item
  In Class: Blank Census forms, 1790-1940; and completed forms for
  Greenleaf St., Boston, 1940 Census.
\end{itemize}

\subsubsection{First Paper Assignments handed out October 3.}

\subsection{Week 6: Sciences of Management}

\subsubsection{October 7: Taylorism}

\begin{itemize}
\item
  \cite{taylor_principles_2006}
\item
  In class: punchcards and sorting.
\end{itemize}

\subsubsection{October 10: Fordism}

\begin{itemize}
\item
  \cite[Chapters 5 and 6]{meyer_five_1981}
\item
  In class: \cite[First fifteen minutes]{chaplin_modern_1936}
\end{itemize}

\subsection{Weeks 7 and 8: A Culture of Data}

(No class October 14/Columbus Day)

\textbf{First Papers Due Wed. October 16 at noon over Blackboard}

\subsubsection{October 17: Meritocracy}

\begin{itemize}
\item
  \cite[pp. 1-173]{lemann_big_1999}
\item
  In class: \cite[Book III, 412b-415c]{plato_republic_1968}
\end{itemize}

\subsection{Week 8: A culture of data: Part II}

\subsubsection{October 21: Quantified Selves}

\begin{itemize}
\item
  \cite[Introduction, Chapters 5 and 6, and Epilogue]{igo_averaged_2007}
\item
  In class: \cite[157-159]{foucault_history_1978}
\end{itemize}

\subsubsection{October 24: Archival Trip}

\begin{itemize}
\item
  Meet in University Archives for backstage tour and to see sample
  holdings.
\end{itemize}

\subsection{Week 9: Early Computers}

\subsubsection{October 28: Imagining Computers}

\begin{itemize}
\item
  \cite{bush_as_1945}
\item
  \cite{bush_memex_1991}
\end{itemize}

\subsubsection{October 31: Gender and Computing}

\begin{itemize}
\item
  \emph{Cosmpolitan,} ``The Computer Girls,'' 1967
\item
  \cite{light_when_1999}
\item
  Film selections in class: \cite{lang_desk_1957}
\end{itemize}

\subsection{Week 10: Computers in the Mainstream}

\subsubsection{November 4: The Mainframe Age}

\begin{itemize}
\item
  \cite[34-69,254-274]{miller_assault_1971}
\end{itemize}

\subsubsection{November 7: Computer Workers}

\begin{itemize}
\item
  \cite{kidder_soul_1981}
\end{itemize}

\subsection{Week 11: Personal Computing}

(No class November 11/Veterans' Day)

\textbf{Data Exploration papers due November 13.}

\subsubsection{November 14: Personal computing}

\begin{itemize}
\item
  Ted M. Lau, ``Total Kitchen Information System'', \emph{Byte
  Magazine}, 1977
\item
  \cite[Introduction; Chapters 1, 2, and 3]{berners-lee_weaving_1999}
\end{itemize}

\subsection{Week 12: The Age of Google}

\subsubsection{November 18: Information Overload revisited}

\begin{itemize}
\item
  \cite[ "After the Deluge"]{gleick_information:_2011}
\item
  \cite[Chapter 2]{vaidhyanathan_googlization_2011}
\end{itemize}

\subsubsection{November 21: The emergence of ``Big Data''}

\begin{itemize}
\item
  The End of Theory, Wired Magazine, 2010 (Read the Introduction and
  pick two fields you are interested in to discuss in class)
\item
  \href{http://norvig.com/chomsky.html}{The Norvig-Chomsky Debate}
\end{itemize}

\subsection{Week 13: Surveillance and the state}

\subsubsection{November 25}

\begin{itemize}
\item
  \cite{macaskill_nsa_2013}
\item
  \cite{gellman_nsa_2013}
\item
  \cite{doctorow_lockdown:_2012}
\end{itemize}

(No class November 28/Thanksgiving)

\subsection{Week 14: Wrapup}

\subsubsection{December 4: Reflections}

\begin{itemize}
\item
  \href{http://www.youtube.com/watch?v=5yB3n9fu-rM}{Watch Edward
  Snowden's interview with Glenn Greenwald}
\item
  (In class)
  \cite[ "Panopticism," pp. 200-204]{foucault_discipline_1977}
\end{itemize}

\subsubsection{Optional Reading for catchup discussion posts}

\begin{itemize}
\item
  (Optional catchup blackboard posts after class)
  \href{http://www.thisamericanlife.org/radio-archives/episode/493/picture-show?act=1}{Listen
  to ``Photo Op,'' \emph{This American Life} Episode 493}
\end{itemize}

\subsection{Final Projects Due December 10}


% Include all texts in bibliography database
\nocite{*}

% But skip any with keyword ``supplemental''
\printbibliography[notkeyword=supplemental,title=Full Citations for Readings]

\vfill
\footnotesize
%\section{Acknowledgments}

Thanks.

\emph{Sample license text:} This syllabus is available for duplication
or modification for other courses and non-commercial uses under a
\href{http://creativecommons.org/licenses/by-nc/3.0/}{CC BY-NC 3.0}
license. Acknowledgment with attribution is requested.

\clearpage

\section{Grading guidelines}

\subsubsection{Written Work}

An \textbf{A} or \textbf{A-} thesis, paper, or exam is one that is good
enough to be read aloud in a class. It is clearly written and
well-organized. It demonstrates that the writer has conducted a close
and critical reading of texts, grappled with the issues raised in the
course, synthesized the readings, discussions, and lectures, and
formulated a perceptive, compelling, independent argument. The argument
shows intellectual originality and creativity, is sensitive to
historical context, is supported by a well-chosen variety of specific
examples, and, in the case of a research paper, is built on a critical
reading of primary material.

A \textbf{B+} or \textbf{B} thesis, paper, or exam demonstrates many
aspects of A-level work but falls short of it in either the organization
and clarity of its writing, the formulation and presentation of its
argument, or the quality of research. Some papers or exams in this
category are solid works containing flashes of insight into many of the
issues raised in the course. Others give evidence of independent
thought, but the argument is not presented clearly or convincingly.

A \textbf{B-} thesis, paper, or exam demonstrates a command of course or
research material and understanding of historical context but provides a
less than thorough defense of the writer's independent argument because
of weaknesses in writing, argument, organization, or use of evidence.

A \textbf{C+}, \textbf{C}, or \textbf{C-} thesis, paper, or exam offers
little more than a mere a summary of ideas and information covered in
the course, is insensitive to historical context, does not respond to
the assignment adequately, suffers from frequent factual errors, unclear
writing, poor organization, or inadequate primary research, or presents
some combination of these problems.

Whereas the grading standards for written work between A and C- are
concerned with the presentation of argument and evidence, a paper or
exam that belongs to the D or F categories demonstrates inadequate
command of course material.

A \textbf{D} thesis, paper, or exam demonstrates serious deficiencies or
severe flaws in the student's command of course or research material.

An \textbf{F} thesis, paper, or exam demonstrates no competence in the
course or research materials. It indicates a student's neglect or lack
of effort in the course.

\subsubsection{Discussions and Seminars}

A student who receives an \textbf{A} for participation in discussion in
discussions or seminars typically comes to every class with questions
about the readings in mind. An `A' discussant engages others about
ideas, respects the opinions of others, and consistently elevates the
level of discussion.

A student who receives a \textbf{B} for participation in discussion in
discussions or seminars typically does not always come to class with
questions about the readings in mind. A `B' discussant waits passively
for others to raise interesting issues. Some discussants in this
category, while courteous and articulate, do not adequately listen to
other participants or relate their comments to the direction of the
conversation.

A student who receives a \textbf{C} for discussion in discussions or
seminars attends regularly but typically is an infrequent or unwilling
participant in discussion.

A student who fails to attend discussions or seminars regularly and
adequately prepared for discussion risks the grade of \textbf{D} or
\textbf{F}.

--Taken from
\href{http://www.princeton.edu/history/undergraduate/grading_practices/}{the
department of history at Princeton University.}


\end{document}

